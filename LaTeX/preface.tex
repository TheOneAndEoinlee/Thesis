\subsection{Elementary Cellular Automata Formalism}
\[
\begin{array}{lll}
\text{1. State Space:} & S = \{0, 1\} \\
\\
\text{2. Neighborhood Configuration:} & N \\
& N = (N_{-1}, N_0, N_1) \quad \text{where} \quad N_{-1}, N_0, N_1 \in S \\
\\
\text{3. Rule Function:} & f: S^3 \to S \\
\\
\text{4. Rule Set:} & R \\
\\
\text{5. Cube Domain:} & D \subset \mathbb{R}^3 \\
& \text{Each vertex directly corresponds to a neighborhood configuration } N, \text{ and its state is given by } f(N). \\
\\
\text{6. Separating Planes:} & P \\
& \text{Defined by a single normal vector } \mathbf{n} \text{ and different offsets } \{d_1, d_2, \ldots, d_n\}. \\
\\
\text{7. Domain Classification Function:} & \Delta: D \to \{0, 1, 2, 3\} \\
& \Delta(x) = \sum_{i=1}^{n} H(n_x \cdot x_x + n_y \cdot x_y + n_z \cdot x_z - d_i) \\
& H(z) = 
\begin{cases} 
0 & \text{if } z < 0 \\
1 & \text{if } z \geq 0 
\end{cases}
\end{array}
\]

\subsection{Wolfram Numbering Scheme for ECA}
\label{sec:Wolfram Numbering Scheme for ECA}
In the Wolfram numbering scheme for Elementary Cellular Automata (ECA), the rule set \( R \) can be uniquely identified by a single integer, which is the binary representation of the output states for all possible neighborhood configurations. For Rule 110, the binary representation is formed by considering all 8 possible 3-cell neighborhood configurations, starting from \( 111 \) down to \( 000 \).

For example, in Rule 110, the corresponding output states for these configurations are \( 01101110 \). Here's how it maps:

\[
\begin{array}{ccc}
\text{Neighborhood Configuration} & \text{Output State} & \text{Binary Position (b)} \\
\hline
111 & 0 & b_7 \\
110 & 1 & b_6 \\
101 & 1 & b_5 \\
100 & 0 & b_4 \\
011 & 1 & b_3 \\
010 & 1 & b_2 \\
001 & 1 & b_1 \\
000 & 0 & b_0 \\
\end{array}
\]

So, the Wolfram number for Rule 110 is obtained by reading the output states from \( b_7 \) to \( b_0 \) as a binary number: \( 01101110_2 = 110_{10} \).

